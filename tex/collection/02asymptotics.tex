% reset section counter
\setcounter{section}{0}

%\metadata{lecture ID}{Your names}{date}
\metadata{2}{Alexander Ke and Trenton Chang}{Jan 13th, 2021}

In this chapter, we use an asymptotic approach (i.e assuming number of training samples $n \to \infty$) to achieve a bound on the ERM. We then instantiate these results to the case where the loss function is the maximum likelihood  and discuss the limitations of asymptotics. (In future chapters we will assume finite $n$ and provide a non-asymptotic analysis.)

\sec{Asymptotics of empirical risk minimization}

For the asymptotic analysis of ERM, we would like to prove that excess risk is bounded as shown below:
\al{
    L(\hat{\theta}) - \argmin_{\theta \in \Theta} L(\theta) \leq \frac{c}{n} + o\left(\frac{1}{n}\right). 
    \label{lec1:eqn:erm-bound}
}
Here $c$ is a problem dependent constant that does not depend on $n$, and $o(1/n)$ hides all dependencies except $n$. The equation above shows that as we have more training data (i.e as $n$ increases) the excess risk of ERM decrease at the rate of $\frac{1}{n}$.

Let $\{(x^{(1)},y^{(1)}), \cdots, (x^{(n)},y^{(n)})\}$ be the training data and let $\cH = \{ h_\theta: \theta \in \R^p \}$ be the parameterized family of hypothesis functions. Let the ERM minimizer be $\hat{\theta}$ as defined in Equation~\eqref{lec1:eqn:erm}. Let $\theta^{*}$ be the minimizer
of the population risk $L$, i.e. $\theta^{*} = \argmin_\theta L(\theta)$. The theorem below quantifies the excess risk $L(\hat{\theta}) - L(\theta^{*})$:

\begin{theorem}[Informally stated]
Suppose that (a) $\hat{\theta}  \overset{p}{\to} \theta^{*}$ as $n \to \infty$ (i.e consistency of $\hat{\theta}$), (b) $\nabla^{2}L(\theta^{*})$ is full rank, and  (c) other appropriate regularity conditions hold. 
\tnoteimp{Perhaps add some footnote explaining $\overset{p}{\to}$ and $\overset{d}{\to}$ briefly and provide links to wikepdia page; and say these concepts are not essential for this course.}
Then,
\begin{enumerate}
    \item $\sqrt{n} (\hat{\theta} - \theta^{*}) = O_P(1)$, i.e. for every $\epsilon > 0$, there is an $M$ such that $\sup_n \bbP (\| \sqrt{n} (\hat{\theta} - \theta^{*}) \|_2 > M) < \epsilon$. (This means that the sequence $\{ \sqrt{n} (\hat{\theta} - \theta^{*}) \}$ is ``bounded in probability".)
    
    \item  $\sqrt{n}(\hat{\theta}-\theta^{*}) \overset{d}{\to} \mathcal{N} \left(0, (\nabla^{2}L(\theta^{*}))^{-1}Cov(\ell((x,y), \theta^*)) (\nabla^{2}L(\theta^{*}))^{-1} \right)$.
     \item $n (L(\hat{\theta}) - L(\theta^{*})) = O_P(1)$.
    \item $n (L(\hat{\theta}) - L(\theta^{*})) \overset{d}{\to} \frac{1}{2} ||S||_{2}^{2}$ where $S \sim \mathcal{N} \left(0, (\nabla^{2}L(\theta^{*}))^{-1/2}Cov(\ell((x,y), \theta^*)) (\nabla^{2}L(\theta^{*}))^{-1/2} \right)$.
    \item $\lim_{n \to \infty} \E \left[ n (L(\hat \theta) - L(\theta^*)) \right] = \frac12 \tr\left( \nabla^2 L(\theta^*)^{-1} \Cov(\nabla \ell ((x, y), \theta^*) \right)$.
\end{enumerate}
\label{lec1:thm:asymp}
\end{theorem}
\textbf{Remark:} In the theorem above, Parts 1 and 3 only show the rate or order of convergence, while Parts 2 and 4 define the limiting distribution for the random variables.

Theorem \ref{lec1:thm:asymp} is a powerful conclusion because once we know that $\sqrt{n}(\hat \theta  - \theta^*)$ is (asymptotically) Gaussian, we can easily work out the distribution of the excess risk. If we believe in our assumptions and $n$ is large enough such that we can assume $n \to \infty$, this allows us to analytically determine quantities of interest in almost any scenario (for example, if our test distribution changes). The key takeaway is that our parameter error $\hat{\theta} - \theta^*$ decreases in order $1/\sqrt{n}$ and the excess risk decreases in order $1/n$. \ttodo{say the $1/n$ rates do rely on the assumption that $L$ is twice differentiable}

\subsec{Key ideas of proofs} 

We will prove the theorem above by applying the following main ideas:
\begin{enumerate}
    \item Obtain an expression for the excess risk by Taylor expansion of the derivative of the empirical risk $\nabla \hatL(\theta)$ around $\theta^{*}$.
    \item By the law of large numbers, we have that $\hatL(\theta) \overset{p}{\to} L(\theta)$, $\nabla\hatL(\theta) \overset{p}{\to} \nabla L(\theta)$   and  $\nabla^{2}\hat{L}(\theta) \overset{p}{\to} \nabla^{2} L(\theta)$ as $n \to \infty$.
    
    \item Central limit theorem (CLT).
\label{ideas}
\end{enumerate}
 
First, we state the CLT for i.i.d. means and a lemma that we will use in the proof.

\begin{theorem}[Central Limit Theorem] \label{lec1:thm:CLT}
Let $X_1, \cdots, X_n$, be i.i.d. random variables, where $\widehat{X}=\frac{1}{n} \sum_{i=1}^{n} X_i$ and the covariance matrix $\Sigma$ is finite. Then, as $n \to \infty$ we have
\begin{enumerate}
    \item $\widehat{X} \overset{p}{\to} \E[X]$, and
    \item $\sqrt{n} (\widehat{X}-\E[X]) \overset{d}{\to} \mathcal{N}(0,\Sigma)$. In particular, $\sqrt{n} (\widehat{X}-\E[X]) = O_P(1)$.
\end{enumerate}
\end{theorem}

\begin{lemma}\label{lec1:lem:dist}
\quad\quad
    \begin{enumerate}
        \item If $Z \sim N(0, \Sigma)$ and $A$ is a deterministic matrix, then $AZ \sim N(0, A \Sigma A^T)$.
        
        \item If $Z \sim N(0, \Sigma^{-1})$ and $Z \in \bbR^p$, then $Z^T \Sigma Z \sim \chi^2(p)$, where $\sim \chi^2(p)$ is the chi-squared distribution with $p$ degrees of freedom.
    \end{enumerate}
\end{lemma}

\subsec{Main proof}

Let us start with heuristic arguments for Parts 1 and 2. First, note that by definition, the gradient of the empirical risk at the empirical risk minimizer, $\nabla \hatL(\hat{\theta})$, is equal to $0$. From the Taylor expansion of $\nabla \hatL$ around $\theta^*$, we have that 
\begin{align}
    0 = \nabla \hatL(\hat{\theta}) = \nabla \hatL(\theta^*) + \nabla^2 \hatL(\theta^*)(\hat{\theta} - \theta^*) + O(\|\hat{\theta} - \theta^*\|^2_2).
\end{align}

Rearranging, we have

\al{
 \hat{\theta}-\theta^{*} = -(\nabla^{2}\hatL(\theta^{*}))^{-1} \nabla \hatL(\theta^{*}) + O(||\hat{\theta}-\theta^{*}||_{2}^{2}). \label{lec1:eqn:branch}} 

Multiplying by $\sqrt{n}$ on both sides,
 \al{
\sqrt{n} (\hat{\theta}-\theta^{*}) &= -(\nabla^{2}\hatL(\theta^{*}))^{-1} \sqrt{n} (\nabla \hatL(\theta^{*})) + O(||\hat{\theta}-\theta^{*}||_{2}^{2})) \\
&\approx -(\nabla^{2}\hatL(\theta^{*}))^{-1} \sqrt{n} (\nabla \hatL(\theta^{*})). \label{lec1:eqn:interm}}

 
Applying the Central Limit Theorem (Theorem~\ref{lec1:thm:CLT}) using $X_i = \nabla \ell ((x^{(i)}, y^{(i)}), \theta^*)$ and $\widehat{X} = \nabla \hatL(\theta^*)$, and noticing that $\E[\nabla \hatL(\theta^{*})] = \nabla L(\theta^{*})$, we have
 \al{ \sqrt{n} (\nabla \hatL(\theta^{*}) - \nabla L(\theta^{*})) \overset{d}{\to} \mathcal{N}(0,Cov(\ell((x, y), \theta^{*}))).} 
 
Note that $\nabla L(\theta^{*}) = 0$ because $\theta^{*}$ is the minimizer of  $L$, so $\sqrt{n} (\nabla \hatL(\theta^{*})) \overset{d}{\to} \mathcal{N}(0,Cov(\ell((x, y), \theta^{*})))$. By the law of large numbers, $\nabla^2 \hatL(\theta^*) \stackrel{p}{\rightarrow} \nabla^2 L(\theta^*)$. Applying these results to \eqref{lec1:eqn:interm} (together with an application of Slutsky's theorem),
\al{
\sqrt{n} (\hat{\theta}-\theta^{*}) &\overset{d}{\to} \nabla^{2}L(\theta^{*})^{-1} \mathcal{N}(0,Cov(\ell((x,y),\theta^{*})) \\
&\stackrel{d}{=} \mathcal{N} \left( 0,\nabla^{2}L(\theta^{*})^{-1}Cov(\ell((x,y), \theta^{*})) \nabla^{2}L(\theta^{*})^{-1} \right),
}

where the second step is due to Lemma~\ref{lec1:lem:dist}. This proves Part 2 of Theorem~\ref{lec1:thm:asymp}.

Part 1 follows directly from Part 2 by the following fact: If $X_n \stackrel{d}{\rightarrow} P$ for some probability distribution $P$, then $X_n = O_P(1)$.

We now turn to proving Parts 3 and 4. Using a Taylor expansion of $L$ with respect to $\theta$ at $\theta^*$, we find
\begin{equation}
L(\hat \theta) = L(\theta^*) 
+ \langle \nabla L(\theta^*), \hat \theta - \theta^* \rangle 
+ \frac12 \langle \hat \theta - \theta^*, \nabla^2 L(\theta^*) (\hat \theta - \theta^*) \rangle + o(\|\hat \theta - \theta^*\|_2^2).
\end{equation}
Since $\theta^*$ is the minimizer of the population risk $L$, we know that $\nabla L(\theta^*) = 0$ and the linear term is equal to 0. Rearranging and multiplying by $n$, we can write
\begin{align}
n (L(\hat \theta) - L(\theta^*)) &= \frac{n}{2} \langle \hat \theta - \theta^*, \nabla^2 L(\theta^*) (\hat \theta - \theta^*) \rangle + o(\|\hat \theta - \theta^*\|_2^2) \\
&\approx \frac12 \langle \sqrt n(\hat \theta - \theta^*), \nabla^2 L(\theta^*) \sqrt n (\hat \theta - \theta^*) \rangle \\
&= \frac12 \left\|\nabla^2 L(\theta^*)^{1/2} \sqrt n(\hat \theta - \theta^*) \right\|_2^2,
\end{align}

where the last equality follows by the fact that for any vector $v$ and square matrix $A$ of appropriate dimensions, the inner product $\langle v, Av\rangle = v^T Av = \lVert A^{1/2}v \rVert_2^2$. Let $S = \nabla^2 L(\theta^*)^{1/2} \sqrt n(\hat \theta - \theta^*)$, i.e. the random vector inside the norm. By Part 2, we know the asymptotic distribution of $\sqrt n(\hat \theta - \theta^*)$ is Gaussian. Thus as $n \to \infty$, $n (L(\hat \theta) - L(\theta^*)) \overset d \to \frac12 \|S\|_2^2$ where
\begin{align}
    S &\sim \nabla^2 L(\theta^*)^{1/2} \cdot \cN \left(0, \nabla^2 L(\theta^*)^{-1} \Cov(\nabla \ell ((x, y), \theta^*) \nabla^2 L(\theta^*)^{-1} \right) \\
    &\stackrel{d}{=} \cN \left(0, \nabla^2 L(\theta^*)^{-1/2} \Cov(\nabla \ell ((x, y), \theta^*) \nabla^2 L(\theta^*)^{-1/2} \right).
\end{align}

This proves Part 4, and Part 3 follows directly from the definition of the $O_P$ notation.

Finally, for Part 5, using the fact that the trace operator is invariant under cyclic permutations, the fact that $\E [S] = 0$, and some regularity conditions,
\begin{align}
    \lim_{n \to \infty} \E \left[ n (L(\hat \theta) - L(\theta^*)) \right] &= \frac12 \E\left[ \|S\|_2^2 \right] = \frac12 \E \left[ \tr(S^\top S) \right] \\
    &= \frac12 \E \left[ \tr(S S^\top) \right]  = \frac12 \tr \left(\E[S S^\top] \right) \\
    &= \frac12 \tr \left( \Cov(S) \right) \\
    &= \frac12 \tr\left( \nabla^2 L(\theta^*)^{-1} \Cov(\nabla \ell ((x, y), \theta^*) \right).
\end{align}

\subsec{Well-specified case}

Theorem \ref{lec1:thm:asymp} is powerful because it is general, avoiding any assumptions of a probabilistic model of our data. However in many applications, we assume a model of our data and we define the log-likelihood with respect to this model. Formally, suppose that we have a family of probability distributions $P_\theta$, parameterized by $\theta \in \Theta$, such that $P_{\theta_*}$ is the true data-generating distribution. This is known as the well-specified case. To make the results of Theorem \ref{lec1:thm:asymp} more applicable, we derive analogous results for this well-specified case in Theorem \ref{lec2:thm:applied}.

\begin{theorem}
\label{lec2:thm:applied}
    In addition to the assumptions of Theorem~\ref{lec1:thm:asymp}, suppose there exists a parametric model $P(y \mid x; \theta)$, $\theta \in \Theta$, such that $\{ y\sp{i} \mid x\sp{i} \}_{i=1}^n \sim P( y\sp{i} \mid x\sp{i} ; \theta_*)$ for some $\theta_* \in \Theta$. Assume that we performing maximum likelihood estimation (MLE), i.e. our loss function is the negative log-likelihood $\ell((x\sp{i}, y\sp{i}), \theta) = - \log P( y\sp{i} \mid x\sp{i} ; \theta)$. As before, let $\hat\theta$ and $\theta^*$ denote the minimizers of empirical risk and population risk respectively. Then
    \al{
    \label{lec2:eqn:applied1}
        \theta^* = \theta_*,
    }
    \al{
    \label{lec2:eqn:applied2}
        \E \left[ \nabla \ell ((x, y), \theta^*) \right] = 0,
    }
    \al{
    \label{lec2:eqn:applied3}
        \Cov \left( \nabla \ell ((x, y), \theta^*) \right) = \nabla^2 L(\theta^*), \text{ and}
    }
    \al{
    \label{lec2:eqn:applied4}
        \sqrt n (\hat \theta - \theta^*) \overset d \to \cN (0, \nabla^2 L(\theta^*)^{-1}).
    }
\end{theorem}

\textbf{Remark 1:} You may also have seen \eqref{lec2:eqn:applied4} in the following form: under the maximum likelihood estimation (MLE) paradigm, the MLE is asymptotically efficient in the Cramer-Rao lower bound. That is, the parameter error of the MLE estimate converges in distribution to $\mathcal{N}(0, \mathcal{I}(\theta)^{-1})$, where $\mathcal{I}(\theta)$ is the Fisher information matrix (in this case, equivalent to the risk Hessian $\nabla^2 L(\theta^*)$)~\cite{rice2006mathematical}.

\textbf{Remark 2:} \eqref{lec2:eqn:applied3} is also known as Bartlett's identity~\cite{percynotes}.

Although the proofs were not presented in live lecture, we include them here.

\begin{proof}
From the definition of the population loss,
\begin{align}
    L(\theta) &= \E \left[ \ell((x\sp{i}, y\sp{i}), \theta) \right]\\
    &= \E \left[ - \log P(y \mid x; \theta) \right] \\
    &= \E \left[ - \log P(y \mid x; \theta) + \log P(y \mid x; \theta_*) \right] + \E \left[ - \log P(y \mid x; \theta_*) \right] \\
    &= \E \left[ \log \frac{P(y \mid x; \theta_*)}{P(y \mid x; \theta)} \right] + \E \left[ - \log P(y \mid x; \theta_*) \right].
\end{align}
Notice that the second term is a constant which we will express as $\cH(y \mid x; \theta_*)$. We expand the first term using the tower rule (or law of total expectation):
\begin{align}
    L(\theta) &= \E \left[ \E \left[ \log \frac{P(y \mid x; \theta_*)}{P(y \mid x; \theta)} \biggr\vert x \right] \right] + \cH(y \mid x; \theta_*).
\end{align}
The term in the expectation is just the KL divergence between the two probabilities, so 
\begin{align}
    L(\theta) &= \E \left[ \KL \left( y \mid x; \theta_* \| y \mid x; \theta \right) \right] + \cH(y \mid x; \theta_*) \\
    &\geq \cH(y \mid x; \theta_*),
\end{align}
since KL divergence is always non-negative. Since $\theta_*$ makes the KL divergence term 0, it minimizes $L(\theta)$ and so $\theta_* \in \argmin_\theta L(\theta)$. However, the minimizer of $L(\theta)$ is unique because of consistency, so  we must have $\argmin_\theta L(\theta) = \theta^*$ which proves (\ref{lec2:eqn:applied1}).

For \eqref{lec2:eqn:applied2}, recall $\nabla L(\theta^*) = 0$, so we have
\begin{equation}
0 = \nabla L(\theta^*) = \nabla \E \left[ \ell((x\sp{i}, y\sp{i}), \theta^*) \right] = \E \left[ \nabla \ell((x\sp{i}, y\sp{i}), \theta^*) \right],
\end{equation}
where we can switch the gradient and expectation under some regularity conditions.

To prove \eqref{lec2:eqn:applied3}, we first expand the RHS using the definition of covariance and express the marginal distributions as integrals:
\begin{align}
    \Cov \left( \nabla \ell ((x, y), \theta^*) \right) &= \E \left[ \nabla \ell ((x, y), \theta^*) \nabla \ell ((x, y), \theta^*)^\top \right] \\
    &= \int P(x) \left( \int P(y \mid x; \theta^*) \nabla \log P( y\sp{i} \mid x\sp{i} ; \theta^*) \nabla \log P( y\sp{i} \mid x\sp{i} ; \theta^*)^\top dy \right) dx \\
    &= \int P(x) \left( \int \frac{\nabla P(y \mid x; \theta^*) \nabla P(y \mid x; \theta^*)^\top}{P(y \mid x; \theta^*)}dy \right) dx.
\end{align}
Now we expand the LHS using the definition of the population loss and differentiate repeatedly:
\begin{align}
    \nabla^2 L(\theta^*) &= \E \left[ \nabla^2 \log P(y \mid x; \theta^*) \right] \\
    &= \int P(x) \left( \int - \nabla^2 P(y \mid x; \theta^*) + \frac{\nabla P(y \mid x; \theta^*) \nabla P(y \mid x; \theta^*)^\top}{P(y \mid x; \theta^*)}dy  \right) dx.
\end{align}
Note that we can express 
\begin{equation} \int \nabla^2 P(y \mid x; \theta^*) dy = \nabla^2 \int P(y \mid x; \theta^*) dy = \nabla 1  = 0 \end{equation}
so we find
\begin{equation} \nabla^2 L(\theta^*) = \int P(x) \left( \int \frac{\nabla P(y \mid x; \theta^*) \nabla P(y \mid x; \theta^*)^\top}{P(y \mid x; \theta^*)}dy \right) dx = \Cov \left( \nabla \ell ((x, y), \theta^*) \right). \end{equation}

Finally, \eqref{lec2:eqn:applied4} follows directly from Part 2 of Theorem~\ref{lec1:thm:asymp} and \eqref{lec2:eqn:applied3}.
\end{proof}

Using similar logic to our proof of Part 4 and 5 of Theorem~\ref{lec1:thm:asymp}, we can see that $n (L(\hat \theta) - L(\theta^*)) \overset d \to \frac12 \|S\|_2^2$ where $S \sim N(0, I)$. Since a chi-squared distribution with $p$ degrees of freedom is defined as a sum of the squares of $p$ independent standard normals, it quickly follows that $2n (L(\hat \theta) - L(\theta^*)) \sim  \chi^2(p)$, where $\theta \in \R^p$ and $n \to \infty$. We can thus characterize the excess risk in this case using the properties of a chi-squared distribution:

\al{
    \lim_{n \to \infty} \E \left[ L(\hat \theta) - L(\theta^*) \right] = \frac{p}{2n}.
}

\sec{Limitations of asymptotic analysis}

One limitation of asymptotic analysis is that our bounds often obscure dependencies on higher order terms. As an example, suppose we have a bound of the form
	\al{
		\frac{p}{2n} + o\left(\frac{1}{n}\right).
		\label{lec2:eqn:spicy_bound}
	}
(Here $o(\cdot)$ treats the parameter $p$ as a constant as $n$ goes to infinity.) 
We have no idea how large $n$ needs to be for asymptotic bounds to be ``reasonable." Compare two possible versions of \eqref{lec2:eqn:spicy_bound}: 
\begin{align}
    \frac{p}{2n} + \frac{1}{n^2} \quad \text{vs.} \quad \frac{p}{2n} + \frac{p^{100}}{n^2}.
\end{align}
Asymptotic analysis treats both of these bounds as the same, hiding the polynomial dependence on $p$ in the second bound. Clearly, the second bound is significantly more data-intensive than the first: we would need $n > p^{50}$ before $\frac{p^{100}}{n^2}$ is less than one. Since $p$ represents the dimensionality of the data, this may be an unreasonable assumption.

This is where non-asymptotic analysis can be helpful. Whereas asymptotic analysis uses large-sample theorems such as the central limit theorem and the law of large numbers to provide convergence guarantees, non-asymptotic analysis relies on concentration inequalities to develop alternative techniques for reasoning about the performance of learning algorithms.

