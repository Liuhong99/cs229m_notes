\sec{Deep Neural Nets (via Covering Number)}\label{sec:deep_nets}
The previous section \tnoteimp{refer to the section using references}discuss how strong our bounds on covering number need to be in order to get a useful result. 
Here we describe some situations in which we know how to obtain these covering number bounds for concrete models such as linear models and neural networks. 

\subsec{Preparation: covering number for linear models}
First, consider the following covering number bound for linear models:

\begin{theorem}[\cite{zhang2002}] \label{lec9:thm:univariate_rad}
Suppose $x^{(1)}, \cdots, x^{(n)} \in \mathbb{R}^d$ are $n$ data points, and $p, q$ satisfies $1/p + 1/q = 1$ and $2 \le p \le \infty$. Assume that $||x^{(i)}||_p \le C$ for all $i$. Let:
\begin{align}
    \cF_q = \{x \mapsto w^\top x : ||w||_q \le B\}
\end{align}
and let $\rho = L_2(P_n)$. Then, $\log N(\epsilon, \cF_q, \rho) \le \l [\frac{B^2C^2}{\epsilon^2}\r ] \log_2 (2d + 1)$. When $p = 2, q = 2$, we further obtain that:
\begin{align}
    \log N(\epsilon, \cF_2, \rho) \le \l [\frac{B^2C^2}{\epsilon^2} \r ] \log_2 (2 \min (n, d ) + 1)
\end{align}
\end{theorem}
\begin{remark}
Applying \eqref{lec9:eqn:rademacherbound_three} to the covering number bound derived above with $R = B^2C^2$, we conclude that the Rademacher complexity of this class of linear models satisfies
\begin{align}
    R_S(\cF_q) &\le \tilO{\left( \frac{BC}{\sqrt{n}} \right)}.
\end{align} 
We also prove this result without relying on Dudley's theorem in Theorem~\ref{lec7:thm:l2-thm}.
\end{remark}
Next, we consider multivariate linear functions as they are building blocks for multi-layer neural networks. Let $M = (M_1, \cdots, M_n) \in \mathbb{R}^{m \times n}$ and $\norm{M}_{2,1} = \sum_{i = 1}^n \norm{M_i}_2$. Then, $\norm{M^\top}_{2,1}$ denotes the sum of the $\ell_2$ norms of the rows of $M$. 
\begin{theorem}\label{lec9:thm:multivariate_rad}
Let $\cF = \{x \to Wx : W \in \mathbb{R}^{m \times d}, ||W^\top||_{2, 1} \le B\}$ and let $C = \sqrt{\frac{1}{n} \sum_{i = 1}^n ||x^{(i)}||_2^2}$. Then, 
\begin{equation}
\log N(\epsilon, \cF, L_2(P_n)) \le \l [\frac{c^2B^2}{\epsilon^2} \r ] \ln (2dm).
\end{equation}
\end{theorem}
\begin{remark}
    In some sense, Theorem~\ref{lec9:thm:multivariate_rad} arises from treating each dimension of the multivariate problem independently. We can view the linear layer as applying $m$ different linear functions. Explicitly, if $W = \begin{pmatrix} w_1^\top \\ \vdots \\ w_m^\top \end{pmatrix}$ and $Wx = \begin{pmatrix} w_1^\top x \\ \vdots \\ w_m^\top x \end{pmatrix}$, then as we expect, $\norm{W^\top}_{2,1} = \sum \norm{w_i}_2$.
\end{remark}


\subsec{Deep neural networks}
In this lecture, we discuss a bound on the Rademacher complexity of a dense neural network. We set up notation as follows: $W_i$ denotes the linear weight matrix at the $i$-th layer of the neural network, we have a total of $r$ layers, and $\sigma$ is the activation function which is 1-Lipschitz (for example, ReLU, softmax, or sigmoid). If the input is a vector $x$, the neural network's output can be represented as follows:

\begin{align}
    f(x) = W_r\sigma(W_{r-1}\sigma(\cdots \sigma(W_1x)\ldots),
\end{align}
Using this notation, we establish an upper bound on the Rademacher complexity of a dense neural network.

\begin{theorem}[\cite{bartlett2017}]
\label{lec10:thm:dnn_rademacher}
Suppose that $\forall i, \norm{x^{(i)}}_2 \leq c$ and let
\begin{align}
    \cF = \{h_\theta : \norm{W_i}_{\text{op}} \leq \kappa_i, \norm{W_i^T}_{2,1} \leq b_i\}.
\end{align}
Then,
\begin{equation}
    R_S (\cF) \leq \frac{c}{\sqrt{n}} \cdot \underbrace{\left(\prod_{i=1}^r \kappa_i \right)}_{\text{relatively large}} \cdot \underbrace{\left( \sum_{i=1}^r\frac{b_i^{2/3}}{\kappa_i^{2/3}}\right)^{3/2}}_{\text{relatively small}}.
\end{equation}
\end{theorem}

\begin{remark}
    We use $\norm{W}_{\textup{op}}$ to denote the operator norm (or spectral norm) of $W$, and $\norm{W_i^\top}_{2,1}$ denotes the sum of the $l_2$ norms of the rows of $W_i$. We note that $f(x) = Wx$ is Lipschitz with a Lipschitz constant of $\norm{W}_{\textup{op}}$. This is because $\norm{f(x)-f(y)}_2 = \norm{Wx-Wy}_2 \leq \norm{W}_{\textup{op}}\norm{x-y}_2$, since $\norm{W}_{\textup{op}} = \max_{x:\norm{x}_2=1}\norm{Wx}_2$. The second term is relatively small as it is a sum of matrix norms, and so the bound is dominated by the first term, which is a product of matrix norms.
\end{remark}

\begin{remark}
    As a corollary of the above theorem, we also get a bound on the generalization error for the margin loss of the following form:
    \begin{equation}
        \texttt{generalization error} \leq \tilde{O}\left(\frac{1}{\gamma_{\min}} \cdot \frac{1}{\sqrt{n}}\cdot c \cdot \left(\prod_{i=1}^r\norm{W_i}_{\textup{op}} \right) \cdot {\left( \sum_{i=1}^r\frac{\norm{W_i^\top}^{2/3}_{2,1}}{\norm{W_i}_{\textup{op}}^{2/3}}\right)^{3/2}}  \right),
    \end{equation}
    where $\gamma_{\min}$ denotes the margin.
\end{remark}
	
First, we motivate the proof by presenting the main idea, and then work through each part of the proof. The main ideas of the proof can be summarized as follows:
    
\begin{itemize}
    \item At a high level, we want to show that the covering number $N(\epsilon, \cF, \rho)$ for a dense neural network is $\leq \frac{R}{\epsilon^2}$. Proving this would enable us to apply the Localized Dudley Theorem to get a bound on the Rademacher Complexity.
    \item To bound the covering number for a dense neural network, we use $\epsilon$-covers to cover each layer of $f$ separately, and then combine them to prove that there exists an $\epsilon$-cover of the original function $f$. 
    \item To combine the epsilon covers of each layer, we use the Lipschitzness of each layer.
    \item We control and approximate the error propagation that is introduced through discretizing each layer using $\epsilon_i$-coverings in order to get a reasonable final $\epsilon$.
\end{itemize}

As a prelude to the proof, let us think of each layer of $f$ as a separate function $f_i$. Note that $f_i$ corresponds to a matrix multiplication and an activation layer. Thus, $f$ can be written as 
\begin{align}
    f = f_r \circ f_{r-1} \circ \cdots \circ f_1,
\end{align}
where each $f_i$ is $\kappa_i$-Lipschitz. Let us also assume, for simplicity, that $f_i(0) = 0$. 

We now derive the $\epsilon$-covering of $f(x)$ in three steps:
\begin{enumerate}
    \item We take points as input to the $i^{th}$ layer, and devise an $\epsilon_i$ covering of the space spanned by each point passed through the function $f_i$.
    \item We use the $N(\epsilon_i, \cF_i, \rho)$ points of the $\epsilon_i$ covering as an input to the $(i+1)$-th layer, where $\cF_i$ is the function family of the $i$-th layer and $\rho = L_2(p_n)$.
    \item Finally, the $\epsilon_r$ covering (i.e covering of the last layer) will be the $\epsilon$ covering of $f(x)$.
\end{enumerate}

Before stating a critical lemma, we establish some definitions. First, define $c_i$ such that all the inputs to the $i^{th}$ layer, $z_1, z_2, \dots, z_n$ are bounded as $\norm{z_j} \leq c_{i-1}$, and therefore $c_0 = c$. We also assume that there exists a function $g$ such that $\log N (\epsilon_i, \cF_i, \rho) \leq g\left(\epsilon_i, c_{i-1}\right)$. Then, we can state the following lemma:

\begin{lemma}
    There exists an $\epsilon$ cover of $\mathcal{F} = \{f(x) : f(x) = f_r \circ f_{r-1} \circ \cdots \circ f_1\}$, where $\epsilon = \epsilon_r + \kappa_r\epsilon_{r-1} + \cdots + \kappa_r\kappa_{r-1}\dots\kappa_2\epsilon_1$, and
    \begin{align}
    \log N(\epsilon, \cF, \rho) \leq \sum_{i=1}^{r} g\left(\epsilon_i, c_{i-1}\right)
    \end{align}
    \label{lec10:lma:additive_cover}
\end{lemma}

\begin{proof}
We define $\cC_1$ to be an $\epsilon_1$-cover of $\cF_1$. For all $f_1' \in \cC_1$, construct $\cC_{2, f_1'}$ to $\epsilon_2$ cover the set $\cF_2 \circ f_1' = \left\{f_2\left(f_1'\left(X\right)\right) : f_2 \in \cF_{2,f_1'} \right\}$. Then, we define $\cC_2 = \cup_{f_1'\in \cC_1}\cC_{2,f_1'}$, and observe that $\cC_2$ a cover of $\cF_2 \circ \cF_1$ with $\epsilon = \epsilon_1 \times \kappa_2 + \epsilon_2$.

We now note that
\begin{align}
    \log \abs{\cC_{2, f_1'}} \leq g\left(\epsilon_2, c_1\right)
\end{align}
Taking a union over all $f_1'$, 
\begin{align}
    \abs{\cC_{2}} &\leq \abs{\cC_{1}} \exp\left(g\left(\epsilon_2, c_1\right)\right) \\
    \log\abs{\cC_{2}} &\leq \log\abs{\cC_{1}} + g\left(\epsilon_2, c_1\right) \leq g\left(\epsilon_1, c_0\right) + g\left(\epsilon_2, c_1\right)
\end{align}
Similarly, given $\cC_k$, for any $f_k' \circ f_{k-1}' \circ \cdots \circ f_1' \in \cC_k$, we construct a $\cC_{k+1, f_k', \dots, f_1'}$ that is an $\epsilon_{k+1}$-covering of $\cF_{k+1} \circ f_k' \circ \cdots \circ f_1'$. We choose $\cC_{k+1} = \cup_{f_i \in \cC_i, i \leq k} C_{k+1, f_k', \dots, f_1'}$. Then, we have
\begin{align}
    \log \abs{\cC_{k+1}} \leq g\left(\epsilon_{k+1}, c_k\right) + \cdots + g\left(\epsilon_1, c_0\right)
\end{align}
Next, we show that for the above construction, the radius of the cover for $\cF$ is
\begin{align}
    \epsilon = \sum_{i=1}^{r} \left(\epsilon_i \prod_{j=i+1}^{r}\kappa_{j}\right).
\end{align}
If we consider a function composition $f_r \circ \cdots \circ f_1 \in \cF_r \circ \cF_{r-1} \circ \cdots \circ \cF_1$, then we know that there exists an $f_1' \in \cC_1$ such that $\rho\left(f_1, f_1'\right) \leq \epsilon_1$. Similarly, we know there exists $f_{2, f_1'}' \in \cC_{2, f_1'}$ such that $\rho\left(f_2' \circ f_1', f_2\circ f_1' \right) \leq \epsilon_2$. Using the triangle inequality, we can say that 
\begin{align}
   \rho\left(f_2' \circ f_1', f_2 \circ f_1\right) &\leq \rho\left(f_2' \circ f_1', f_2 \circ f_1'\right) + \rho\left(f_2 \circ f_1', f_2 \circ f_1\right) \\ 
   &\leq \epsilon_2 + \rho\left(f_2 \circ f_1', f_2 \circ f_1\right) \\ 
   &\leq \epsilon_2 + \kappa_2 \rho\left(f_1', f_1\right) \\ 
   &\leq \epsilon_2 + \kappa_2\epsilon_1z
\end{align}
Using the same logic for a general $k$, we know that there exists $f_{k,f_1',\dots,f_{k-1}'}' \in \cC_k$ such that
\begin{align}
    \rho\left(f_k' \circ f_{k-1}' \circ \cdots \circ f_1', f_k \circ \cdots \circ f_1\right) &\leq \rho\left(f_k' \circ f_{k-1}'\circ \cdots \circ f_1', f_k \circ f_{k-1}'\circ \cdots \circ f_1' \right) \\ 
    &+ \rho\left(f_k \circ f_{k-1}'\circ f_{k-2}' \circ \cdots \circ f_1', f_k \circ f_{k-1}\circ f_{k-2}' \circ \cdots \circ f_1'\right) \\ &+ \cdots + \rho\left(f_k \circ f_{k-1}\circ \cdots \circ f_2 \circ f_1', f_k \circ f_{k-1}\circ \cdots \circ f_1\right) \\ & \leq \sum_{i=1}^{r} \left(\epsilon_i\prod_{j=i+1}^{r}\kappa_{j}\right)
\end{align}
\end{proof}

\begin{proof}[Proof of Theorem~\ref{lec10:thm:dnn_rademacher}]
We now apply Lemma~\ref{lec10:lma:additive_cover} to dense neural networks. Dense neural networks consist of a composition of layers, where each layer is a linear model composed with a 1-Lipschitz activation. Using Theorem~\ref{lec9:thm:multivariate_rad} along with the property that 1-Lipschitz functions will only contribute a factor of at most $1$ (Lemma~\ref{lec9:lma:talagrand}), the covering number of each layer can be bounded by:
\begin{align}
    g\left(\epsilon_i, c_{i-1}\right) = \tilde{O}\left(\frac{c_{i-1}^2b_i^2}{\epsilon_i^2}\right),
\end{align}
where $c_{i-1}^2$ is the norm of the inputs, $b_i^2$ is $\norm{W_i^T}_{2,1}$, and $\epsilon_i^2$ is the radius. From Lemma~\ref{lec10:lma:additive_cover}, we know that 
\begin{align}
    \log N(\epsilon, \cF, \rho) &= \tilde{O}\left(\sum_{i=1}^{r}\frac{c_{i-1}^2b_i^2}{\epsilon_i^2}\right) 
\end{align}
for
\begin{align}
    \epsilon &= \sum_{i=1}^{r} \left(\epsilon_i \prod_{j=i+1}^{r}\kappa_j\right)
\end{align}

We will now choose $\epsilon_i$ that achieves $\log\left(N\left(\epsilon, \cF, \rho\right)\right) \leq \frac{R}{\epsilon^2}$, where $R$ is some constant independent of $\epsilon$. We motivate this choice of $\epsilon$ using Holder's inequality, which states that
\begin{align}
    \langle a,  b \rangle \leq \|a\|_p \|b\|_q
\end{align}
when $\frac{1}{p} + \frac{1}{q} = 1$. Writing out the vectors $a, b$, we get that 
\begin{align}
    \sum_{i}a_ib_i \leq \left(\sum a_i^p\right)^{\frac{1}{p}}\left(\sum b_i^q\right)^{\frac{1}{q}}
\end{align}

Letting $\alpha_i^2 = c_{i-1}^2b_i^2, \beta_i = \prod_{j=i+1}^{r}\kappa_j$. By Holder's inequality, using $p = 3, q = \frac{3}{2}$, we get
\begin{align}
    \left(\sum_{i=1}^{r}\frac{\alpha_i^2}{\epsilon_i^2}\right)\left(\sum_{i=1}^{r}\beta_i\epsilon_i\right)^2 &\geq \left(\sum_{i=1}^{r}\left(\alpha_i\beta_i\right)^{\frac{2}{3}}\right)^{\frac{3}{2}} \\
    \sum_{i=1}^{r}\frac{\alpha_i^2}{\epsilon_i^2} &\geq \frac{R}{\epsilon^2},
\end{align}
where $R = \left(\left(\sum_{i=1}^{r}\left(c_{i-1}b_i\prod_{j=i+1}^{r}\kappa_j\right)\right)^{\frac{2}{3}}\right)^{\frac{3}{2}}$. We note that equality holds when 
\begin{align}
    \epsilon_i = \left(\frac{c_{i-1}^2b_i^2}{\prod_{j=i+1}^{r}\kappa_j}\right)^{\frac{1}{3}} \cdot \underbrace{\frac{\epsilon}{\left(\sum_{i=1}^{r}\frac{b_i^{\frac{2}{3}}}{\kappa_i^{\frac{2}{3}}}\right)\prod_{i=1}^{r}\kappa_i^{\frac{2}{3}} }}_{\epsilon'} \label{eqn:lec10:holder_eps_defn}
\end{align}
Using this choice of $\epsilon_i$ and letting $\epsilon'$ equal the second factor in \eqref{eqn:lec10:holder_eps_defn} for notational convenience, we know that the log covering number is (up to a constant factor):
\al{
    \sum_{i=1}^r \frac{c_{i-1}^2b_i^2}{\epsilon_i^2} &= \sum_{i=1}^r \frac{c_{i-1}^2b_i^2(\kappa_{i+1}\cdots\kappa_r)^\frac{2}{3}}{c_{i-1}^\frac{4}{3}b_i^\frac{4}{3}(\epsilon')^2} \\
    &= \sum_{i=1}^r (c_{i-1}b_i\kappa_{i+1}\cdots\kappa_r)^\frac{2}{3}\frac{1}{(\epsilon')^2} \\
    &= c^\frac{2}{3}\sum_{i=1}^r \left(\frac{b_i}{\kappa_i}\right)^\frac{2}{3} \prod_{i=1}^r \kappa_i^\frac{2}{3} \frac{\left(c^\frac{2}{3}\left(\sum_{i=1}^r (\frac{b_i}{\kappa_i})^\frac{2}{3} \prod_{i=1}^r \kappa_i^\frac{2}{3}\right)\right)^2}{\epsilon^2} \\
    &= \left(c^\frac{2}{3}\sum_{i=1}^r \left(\frac{b_i}{\kappa_i}\right)^\frac{2}{3} \prod_{i=1}^r \kappa_i^\frac{2}{3}\right)^3\frac{1}{\epsilon^2} \\
    &= c^2\prod_{i=1}^r \kappa_i^2\left(\sum_{i=1}^r \left(\frac{b_i}{\kappa_i}\right)^\frac{2}{3}\right)^3\frac{1}{\epsilon^2}.
}
Since this log covering number is of the form $R / \epsilon^2$, we can apply \eqref{lec9:eqn:rademacherbound_three} and conclude that
\al{
    \mathcal{R}_S(\cF) \lesssim \sqrt\frac{R}{n}
}
Last, plugging in
\al{
    R = c^2\prod_{i=1}^r \kappa_i^2\left(\sum_{i=1}^r \left(\frac{b_i}{\kappa_i}\right)^\frac{2}{3}\right)^3
}
we obtain the desired result
\al{
    \mathcal{R}_S(\cF) \lesssim \frac{c}{\sqrt n}\prod_{i=1}^r \kappa_i\left(\sum_{i=1}^r \left(\frac{b_i}{\kappa_i}\right)^\frac{2}{3}\right)^\frac{3}{2}.
}

\end{proof}
