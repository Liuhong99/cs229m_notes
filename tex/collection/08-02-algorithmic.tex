% reset section counter
%\setcounter{section}{0}

%\metadata{lecture ID}{Your names}{date}
\metadata{14}{Roshni Sahoo and Sarah Wu}{Mar 3rd, 2021}

\subsec{Proof of main result: gradient descent on empirical loss}

Analyzing gradient descsent on the empirical risk $\empL$ is more complicated than analyzing gradient descent on the population risk, so we focus on the case when $\beta^\star$ is $1$-sparse, i.e. $r=1$. (When $r>1$, the main idea is the same but requires some more advanced analysis techniques.) 

Note that in our setup, i.e. when $x^{(1)} \ldots x^{(n)} \iid \mathcal{N}(0, I_{d\times d})$ and when $n\geq \widetilde{\Omega}(r/\delta^2)$, with high probability the data satisfy the $(r, \delta)$-RIP condition. It follows that when $r=1$ and $\delta = \tilO(1/\sqrt{n})$, the data are $(1, \delta)$-RIP. This will allow us to use the lemmas involving the RIP condition for the proof.

We restate the case of $r=1$ in the following theorem.
 
\begin{theorem} \label{lec14:thm:main}
Suppose $\eta \geq \widetilde{\Omega}(1).$ Then, gradient descent on $\empL$ with $t = \Theta \l(\frac{\alpha \log (1/\delta)}{\eta}\r)$ steps satisfies 
\begin{equation}
\Norm{\beta^{t} \odot \beta^{t} - \beta^\star \odot \beta^\star}_{2}^{2} \leq \tilO\l(\frac{1}{\sqrt{n}}\r).
\end{equation} 
\end{theorem}

\begin{remark}
Note that Theorem~\ref{lec14:thm:main} is a slightly weaker version of Theorem~\ref{lec13:thm:non-linear-main} for $r=1$, since the bound on the RHS depends on the number of examples and not the initialization $\alpha$. In Theorem~\ref{lec13:thm:non-linear-main}, we could take $\alpha$ as small as we like to drive the bound to zero; we cannot do this for Theorem~\ref{lec14:thm:main}.
\end{remark}

We proceed to prove Theorem~\ref{lec14:thm:main} with the follow steps:
\begin{enumerate}
\item Computing the gradient update $\nabla \empL$,
\item Dynamics analysis of noise $\zeta_t$, 
\item Dynamics analysis of signal $r_t$, and
\item Putting it all together.
\end{enumerate}

\underline{Computing the gradient update $\nabla \empL$}

WLOG, assume that $\beta^\star = e_{1}.$ We can decompose the gradient descent iterate $\beta^{t}$ as
\begin{equation}
    \beta^{t} = r_{t} \cdot e_{1} + \zeta_{t},
\end{equation}
where $\zeta_t \perp e_1$. The idea is to prove convergence to $\beta^\star$ by showing that (i) $r_{t} \rightarrow 1$ as $t \rightarrow \infty$, and (ii) $\norm{\zeta_{t}}_{\infty} \leq O(\alpha)$ for $t \leq \tilO\big(1/\eta).$ In other words, the \textit{signal} $r_{t}$ converges quickly to $1$ while the \textit{noise} $\zeta_t$ remains small for some number of initial iterations. One may be concerned that it is possible for the noise to amplify after many iterations, but we will not have to worry about this scenario if we can guarantee that $\beta^{t}$ converges to $\beta^\star$ first.

We can compute the gradient of $\empLt$ as follows. Since $y\sp{i} = \langle \beta^\star \odot \beta^\star, x\sp{i} \rangle$ and $\beta^{t} = r_{t}e_{1} + \zeta_{t} = r_{t}\beta^\star + \zeta_{t}$,
\begin{align}
    \nabla \empLt &= \frac{1}{n} \sum_{i=1}^{n} (\langle \beta^{t} \odot \beta^{t}, x\sp{i} \rangle - y\sp{i} ) x\sp{i} \odot \beta^{t} \\
    &= \frac{1}{n} \sum_{i=1}^{n} ( \langle \beta^{t} \odot \beta^{t} - \beta^\star \odot \beta^\star, x\sp{i} \rangle ) x\sp{i} \odot \beta^{t} \\
    &= \frac{1}{n} \sum_{i=1}^{n} \langle r_{t}^{2} \beta^\star \odot \beta^\star + \zeta_{t} \odot \zeta_{t} - \beta^\star \odot \beta^\star, x\sp{i}  \rangle x\sp{i} \odot \beta^{t} \\
    &= \underbrace{\frac{1}{n} \sum_{i=1}^{n} \Big\langle \big(r_{t}^{2} - 1\big) \beta^\star \odot \beta^\star + \zeta_{t} \odot \zeta_{t}, x\sp{i}  \Big\rangle x\sp{i}}_{m_t} \odot \beta^{t}.
\end{align}

To simplify the analysis, we can rearrange some of the terms that are part of the gradient. Define $m_{t} $ such that $\nabla \empLt = m_{t} \odot \beta^{t}.$ Also, let $X = \frac{1}{n} \sum_{i=1}^{n} x\sp{i}(x\sp{i})^\top.$ Then,
\begin{align}
    m_{t} &= \frac{1}{n} \sum_{i=1}^{n} \Big\langle \big(r_{t}^{2} - 1\big) \beta^\star \odot \beta^\star + \zeta_{t} \odot \zeta_{t}, \ x\sp{i}  \Big\rangle \ x\sp{i} \\
    &= \l( \frac{1}{n} \sum_{i=1}^{n} x\sp{i}\big(x\sp{i}\big)^\top \r) \l(r_{t}^{2} - 1\r) \cdot \l(\beta^\star \odot \beta^\star\r) + \l( \frac{1}{n} \sum_{i=1}^{n} x\sp{i}(x\sp{i})^\top \r) \l(\zeta_{t} \odot \zeta_{t}\r) \\
    &= \underbrace{X \big(r_{t}^{2} - 1\big) \cdot \big(\beta^\star \odot \beta^\star\big)}_{\text{part of } u_t} + \underbrace{X \big(\zeta_{t} \odot \zeta_{t}\big)}_{v_t}.
\end{align}

Now, define $u_{t} := (r_{t}^{2} - 1) (\beta^\star \odot \beta^\star) - X (r_{t}^{2} - 1) (\beta_{*} \odot \beta_{*})$ and $v_{t} := X \big(\beta_{t} \odot \beta_{t}\big)$. Then we can rewrite the gradient as

\begin{equation}
    \nabla \empLt = m_{t} \odot \beta^{t} = [(r_{t}^{2} -1) \beta^\star \odot \beta^\star - u_{t} + v_{t}] \odot \beta_{t}. \label{lec14:eqn:emp-gradient}
\end{equation}

Our goal is to show that both $u_t$ and $v_t$ are small, so that $\nabla \empLt$ is close to its population version $\nabla L(\beta^t)$. Observe that $X$ appears in both $u_{t}$ and $v_{t}$. This matrix is challenging to deal with mathematically because it does not have full rank (because $n < d$). Instead, we rely on the RIP condition to reason about the behavior of $X$: the idea is that $X$ behaves like the identity for sparse vector multiplication. Applying Corollary~\ref{lec14:cor:rip}, we can bound $\Norm{u_{t}}_{\infty}$ as
\begin{equation} \label{lec14:eqn:u-inf-norm}
    \Norm{u_{t}}_{\infty} \leq 4\delta \Norm{(r_t^2 - 1)  \beta^\star \odot \beta^\star }_{2} 
    \leq 4\delta ||\beta^\star \odot \beta^\star||_{2} \leq 4\delta.
\end{equation}

(In the second inequality, we assume that $|r_t| < 1$. We can do this because $r_t$ starts out at $\alpha$ which is small; if $r_t \geq 1$, then we are already in the regime where gradient descent has converged.) We can bound $\Norm{v_{t}}_{\infty}$ in a similar manner: since Corollary~\ref{lec14:cor:rip} implies $\Norm{v_t - \zeta_t \odot \zeta_t}_\infty \leq 4\delta \Norm{\zeta_{t} \odot \zeta_{t}}_{2}$,
\begin{align}
    \Norm{v_{t}}_{\infty} &\leq \Norm{\zeta_{t} \odot \zeta_{t}}_{\infty} + 4\delta \Norm{\zeta_{t} \odot \zeta_{t}}_{2} &(\text{by the triangle inequality}) \\
    &\leq \Norm{\zeta_{t}}_{\infty}^{2} + 4\delta \Norm{\zeta_{t} \odot \zeta_{t}}_{1} &(\text{since } \zeta_t \text{ very small}) \\
    &= \Norm{\zeta_{t}}_{\infty}^{2} + 4\delta \Norm{\zeta_{t}}_{2}^{2}. \label{lec14:eqn:v-inf-norm}
\end{align}

Note that the size of $v_t$ depends on the size of the noise $\zeta_t$. Thus, by bounding $\zeta_t$ (e.g. with a small initialization), we can ensure that $v_t$ is also small. (Ensuring bounds on $u_t$ is more difficult because it depends only on $\delta$.) In the next two subsections, we analyze the growth of $\zeta_t$ and $r_t$.

\underline{Dynamics analysis of $\zeta_t$}

First, we analyze the dynamics of the noise $\zeta_t$, which we want to ensure does not grow too fast.

\begin{lemma} \label{lec14:lem:dynamics_noise}
    For all $t\leq 1 / (c\eta\delta)$ with sufficiently large constant $c$, we have
    \begin{equation} \label{lec14:eqn:dynamics_noise}
        \Norm{\zeta_t}_\infty \leq 2\alpha, \quad \quad \Norm{\zeta_t}_2^2 \leq \frac{1}{2}, \quad \quad \text{and} \quad \Norm{\zeta_{t+1}}_\infty \leq \big(1 + O(\eta\delta)\big) \Norm{\zeta_t}_\infty.
    \end{equation}
\end{lemma}
Note that this result is weaker than what we were able to show for the population gradient (exponential growth with a small fixed rate), but we will ultimately show that the growth of the signal will be even faster.

\begin{proof}
Recall that the empirical gradient \eqref{lec14:eqn:emp-gradient} is $\nabla \hat{L}(\beta) = \big[(r_{t}^{2} - 1) \beta^\star \odot \beta^\star - u_{t} + v_{t} \big] \odot \beta^{t}$. Hence, the gradient update to $\beta^{t}$ is

\begin{align}
\beta^{t+1} &= \beta^{t} - \eta \l[\l(r_{t}^{2} - 1\r) \beta^\star \odot \beta^\star - u_{t} + v_{t} \r] \odot \beta^{t} \\
&= \underbrace{\beta^{t} - \eta \l(r_{t}^{2} - 1\r) \beta^\star  \odot \beta^\star \odot \beta^{t}}_{\text{GD update for population loss}} - \eta \l(- u_{t} + v_{t}\r) \odot \beta^{t}. \label{lec14:eqn:gd-update}
\end{align}
    
Recall that $\zeta_{t+1}$ is simply $\beta^{t+1}$ except for the first coordinate (where it has a zero instead of $r_{t+1}$), i.e. $\zeta_{t+1}$ is the projection of $\beta^{t+1}$ onto the subspace orthogonal to $e_1$. Hence,
\begin{align}
\zeta_{t+1} &= \l(I - e_{1} e_{1}^\top\r) \beta^{t+1} \\
&= \l(I - e_{1} e_{1}^\top\r) \cdot \beta^{t} - \eta \l(I - e_{1} e_{1}^\top\r) (v_{t} - u_{t}) \odot \beta^{t} &\text{(by \eqref{lec14:eqn:gd-update}, second term = 0)} \\
&= \zeta_{t} - \eta \l[\l(I - e_{1}e_{1}^{T}\r) (v_{t} - u_{t}) \odot \l(I - e_{1}e_{1}^{T}\r) \beta^{t}\r] &(\text{by distribution law for $\odot$}) \\
&= \zeta_t - \eta \underbrace{\l[ \l(I - e_{1}e_{1}^{T}\r) \l(v_t - u_t\r)\r]}_{\rho_t} \odot \zeta_t.
\end{align}
    
If we define $\rho_t$ such that $\zeta_{t+1} = \zeta_t - \eta \rho_t \odot \zeta_t$, then the growth of $\zeta_t$ is dictated by the size of $\rho_t$. We can bound this as
\begin{equation}
\Norm{\zeta_{t+1}}_{\infty} \leq (1 + \eta \Norm{\rho_{t}}_{\infty}) \Norm{\zeta_{t}}_{\infty}. \label{lec14:eqn:zeta-growth-bd}
\end{equation}

Now, we will prove the lemma by using strong induction on $t$. Suppose that the first two pieces of \eqref{lec14:eqn:dynamics_noise} hold for all iterations up to $t$. We can show that
\begin{align}
\Norm{\rho_{t}}_{\infty} &\leq \Norm{u_{t}}_{\infty} + \Norm{v_{t}}_{\infty}  \\
&\leq 4\delta + \Norm{\zeta_t}_{\infty}^{2} + 4\delta \Norm{\zeta_t}_{2}^{2} &(\text{by \eqref{lec14:eqn:u-inf-norm} and \eqref{lec14:eqn:v-inf-norm}}) \\
&\leq  4\delta + (2\alpha)^2 + 4\delta \cdot \frac{1}{2} &(\text{by the inductive hypothesis})\\
\label{lec14:eqn:diff-inf-norm}
&\leq 8\delta,
\end{align}
where the last step holds because we can take $\alpha$ to be arbitrarily small (e.g. $\alpha \leq \text{poly}(1/n) \leq O(\delta)$). Plugging this into \eqref{lec14:eqn:zeta-growth-bd}, we have
\begin{equation}
\Norm{\zeta_{t+1}}_\infty \leq (1 + 8\eta \delta) \Norm{\zeta_t}_{\infty} = \big(1 + O(\eta\delta)\big) \Norm{\zeta_t}_\infty,
\end{equation}
which proves the third piece of the lemma. Using this piece, we can show that
\begin{equation}
\Norm{\zeta_{t+1}}_{\infty} \leq \l(1 + 8 \eta \delta\r)^{t+1} \Norm{\zeta_{0}}_{\infty} \leq \big(1 + 8\eta \delta\big) ^{1/(c\eta \delta)} \cdot \alpha  \leq 2\alpha
\end{equation}
for a sufficiently large constant $c$, which proves the second piece. Finally, we show that
\begin{equation}
\Norm{\zeta_{t+1}}_{2}^2 \leq \big(1 + 8\eta \delta\big)^{t+1}\Norm{\zeta_{0}}_{2}^2 \leq \big(1 + 8\eta \delta)^{1/(c\eta \delta)} \cdot \alpha \sqrt{d} \leq \frac{1}{2},
\end{equation}
if $\alpha \leq \frac{1}{n^{O(1)}}$, which proves the first piece.

\end{proof}

\underline{Dynamics analysis of $r_t$}

Next, we analyze the dynamics of the signal $r_t$, which we want to show converges to 1.

\begin{lemma} \label{lec14:lem:dynamics_signal}
    For all $t\leq 1 / (c\eta\delta)$ with sufficiently large constant $c$, we have that
    \[ r_{t+1} = \big(1 + \eta\big( 1 - r_t^2 \big) \big) r_t + O\big(\eta\delta\big) r_t. \]
\end{lemma}
Note that the first term on the RHS is $r_{t+1}$ during gradient descent on the population loss, and the second term captures the error.

\begin{proof}
    Recall that the gradient descent update from the empirical gradient~\eqref{lec14:eqn:emp-gradient} is
    \begin{equation}
        \beta^{t+1} = \beta^t - \eta \big[\big(r_{t}^{2} -1\big) \beta^\star \odot \beta^\star - u_{t} + v_{t}\big] \odot \beta_{t}.
    \end{equation} 
    We have that
    \begin{align}
        r_{t+1} &= \big\langle \beta^{t+1}, e_1\big\rangle \\
        &= \big\langle \beta^t, e_1\big\rangle - \eta \big(r_{t}^{2} -1\big)\big\langle \beta^t, e_1\big\rangle - \eta \big\langle v_t-u_t, e_1\big\rangle \big\langle \beta^t, e_1 \big\rangle \\
        &= r_t - \eta \big(r_{t}^{2} -1\big) r_t - \eta \big\langle v_t-u_t, e_1\big\rangle r_t \\
        &= \Big(1 + \eta\big( 1 - r_t^2 \big) \Big) r_t + \eta \big\langle u_t-v_t, e_1\big\rangle r_t
    \end{align}
    so all we need to do is bound the second term as follows:
    \begin{align}
        |\eta \langle v_t - u_t, e_1\rangle r_t| &\leq \eta \cdot r_t \Norm{v_t-u_t}_\infty \\
        &\leq \eta \cdot r_t \cdot 8\delta &(\text{by \eqref{lec14:eqn:diff-inf-norm}}) \\
        &= O(\eta\delta) \cdot r_t.
    \end{align}
\end{proof}

\underline{Putting it all together}
Finally, we return to the proof of Theorem~\ref{lec14:thm:main}. By Lemma~\ref{lec14:lem:dynamics_signal}, we know that as long as $r_t \leq 1/2$ it will grow exponentially fast, since
\begin{equation}
    r_{t+1} \geq \Big(1 + \eta\big(1-r_t^2\big) - O(\eta\delta) \Big) \cdot r_t \geq \bigg(1 + \frac{\eta}{2}\bigg)\cdot r_t.
\end{equation} 
This implies that at some $t_0 = O\Big(\frac{\log (1/\alpha)}{\eta}\Big)$, we'll observe $r_{t_0} > 1/2$ for the first time. Consider what happens after this point.

\begin{itemize}
    \item When $1/2 < r_t \leq 1$, we have that
    \begin{align}
        1 - r_{t+1} &\leq 1 - r_t - \eta \big(1 - r_t^2\big) r_t + O(\eta\delta) \cdot r_t \\
        &\leq 1 - r_t - \frac{\eta \big(1 - r_t^2\big)}{2} + O(\eta\delta) \\
        &\leq 1 - r_t - \frac{\eta \big(1 - r_t\big)}{2} + O(\eta\delta) \\
        &= \bigg(1 - \frac{\eta}{2}\bigg) (1 - r_t) + O(\eta\delta).
    \end{align}
    Thus, we can achieve $1 - r_{t+1} \leq 2 \cdot \frac{O(n\delta)}{\eta/2} = O(\delta)$ in $\Theta\Big(\frac{\log(1/\delta)}{\eta}\Big)$ iterations.
    
    \item When $r_t > 1$, we can show in a similar manner that
    \begin{equation}
        r_{t+1} - 1 \leq (1 - \eta) (r_t - 1) + O(\eta\delta) \leq O(\delta),
    \end{equation} 
    implying that $r_t$ remains very close to 1 after the same order of iterations.
\end{itemize}

This completes the proof of Theorem~\ref{lec14:thm:main}, bounding the number of iterations needed for gradient descent on the empirical loss to converge to $\beta^*$.
\qed 
